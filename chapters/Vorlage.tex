\chapter{Einleitung\label{chap1:Erstes-Kapitel}}

Einleitender Text zum Kapitel.

Eintrag im Abkürzungsverzeichnis: Beschreibung (Bschr.\nomenclature{Bschr}{Beschreibung})

\section{Motivation und Problemstellung\label{sec1.1:Unterpunkt-1}}

Abschnitt 1.

%\begin{center}
%    \begin{figure}[H]
%        \begin{centering}
%            <includegraphics{<image>}>
%            \par
%        \end{centering}
%       \caption[Kurztext für Inhaltsverzeichnis\label{fig1.1:Beispielgrafik}]{Beispiel für eine eingebettete Bilddatei. }
%    \end{figure}
%    \par
%\end{center}

\subsection{Unterabschnitt 1\label{sub1.1.1:Unterpunkt-2}}

Unterabschnitt 1.

\begin{table}[H]
    \subsubsection{\textrm{Testtabelle}}
    \resizebox{\textwidth}{!}{%
        \begin{raggedright}
            \begin{tabular}{|>{\centering}m{1cm}|>{\raggedright}p{3.5cm}|>{\raggedright}m{3.5cm}|>{\raggedright}m{7cm}|>{\raggedright}m{6cm}|}
                \hline 
                \# & Spalte1 & \multirow{1}{3.5cm}{Spalte2} & Spalte3 & Spalte4\tabularnewline
                \hline 
                \hline 
                1 & bla & bla & bla & 
                \begin{itemize}
                    \item bla
                \end{itemize}
            \tabularnewline
            \hline 
            \end{tabular}
        \par
        \end{raggedright}
    }
    \caption[Text für Inhaltsverzeichnis]{Beschreibender Text für die Tabelle. \label{tab1.1:Beispieltabelle}}
\end{table}

\subsection{Unterabschnitt 2}

Unterabschnitt 2.

\begin{lstlisting}[label=lst1.1:Beispiel, caption={[Kurztext fürs Inhaltsverzeichnis]Beispielblock für Code. Syntax-Highlighting etc. kann unter ,,Document`` -> ,,Settings`` -> ,,\protect\LaTeX{} Preamble`` konfiguriert werden.}, captionpos=b]
    public class TodotasksListUnitTest {
        @Test
        public void testReturnDueTasks() throws Exception{
            List<Todotasks> mockToDoTasks = new ArrayList<Todotasks>();
            mockToDoTasks = getTestTodoTasks();
            Date testDueDate = new SimpleDateFormat("MM/dd/yyyy").parse("12/05/2016");
            int dueTodoTasksCount = new TodotasksList().getDueTasks(mockToDoTasks, testDueDate).size();

            Assert.assertEquals(3, dueTodoTasksCount);
        }
    }
\end{lstlisting}
Verweis aufs Listing: \ref{lst1.1:Beispiel}

\subsection{Unterabschnitt 3}

Unterabschnitt 3 zeigt das Verwenden des Glossaries Package.\newline

Für die Verweise auf das die definierten Glossare. \Gls{latex} ist eine Markuplanguage für das Schreiben von wissenschaftlichen Arbeiten. Das \acrfull{emes} ist ein \acrfull{mes} der Firma Eisenmann SE. Ein \acrshort{mes} System, zu deutsch Produktionsleitsystem, agiert zwischen der Unternehmensleitebene und der Kontrollschicht. Ein Beispiel für ein \acrlong{mes} ist das Produktionsleitsystem \acrshort{emes}.