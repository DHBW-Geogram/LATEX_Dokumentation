\chapter{Einführung\label{chap1:Erstes-Kapitel}}

Nachfolgend wird eine Einleitung in die hier vorliegende Projektdokumentation gegeben. Neben einer vorgegebenen Aufgabenstellung und selbstdefinierter Kernidee, wird eine Auflistung aller Projektteilnehmer aufgezeigt.

\section{Aufgabenstellung\label{sec1.1:Unterpunkt-1}}

Zieles dieses Projektes ist die Konzeption und Implementierung einer mobilen App. Jenes Projekt findet im Rahmen der Vorlesung \glqq Entwicklung mobiler Applikationen\grqq{} statt.

\begin{itemize}
    \item Der Use Case sollte kein Spiel sein, sondern eher eine klassische App abbilden.
    \item Der Use Case soll mindestens von zwei der zur Verfügung stehenden Funktionen eines Mobilgeräts nutzen. Beispielsweise
    \begin{itemize}
        \item GPS-Sensor
        \item Neigungssensor
        \item Kamera
        \item Fingerabdruck
        \item NFC
    \end{itemize}
    \item Eine FSK Einstufung der Applikation sollte nicht nötig sein.
\end{itemize}

\section{Projektidee\label{sec1.2:Unterpunkt-2}}

Als Gruppe wurde eine Kernidee für die Erfüllung der Projektanforderungen entworfen. Die mobile Anwendung orientiert sich an der bestehenden mobilen Anwendung \glqq Instagram\grqq{}.

Ziel ist eine Plattform für das Teilen von Bildern und Entdecken von Content, welche sich in der Nähe des Benutzers befinden. Vergleichbar zu Instagram sollen relevante Beiträge in Feeds dargestellt werden. Der große Unterschied zu Instagram ist der Fokus auf die geografischen Hintergrundinformationen der Beiträge. So werden den Benutzern lediglich die Beiträge von anderen Benutzern angezeigt, welche sich in einer festgelegten geografischen Entfernung befinden. Durch die Einbindung und Verwendung von geografischen Informationen, soll die Vermarktung von lokalen Angeboten, Aktivitäten und Sehenswürdigkeiten erleichtert werden.

Durch die große Ähnlichkeit mit Instagram, und der Erweiterung um die Nutzung von GPS-Informationen, für die Darstellung von ortsnahen Feed-Beiträgen, wurde sich einheitlich für den Projektnamen \glqq Geogram\grqq{} entschieden.

Wie in der Aufgabenstellung (\autoref{sec1.1:Unterpunkt-1}) gefordert, beinhaltet die mobile Anwendung \glqq Geogram\grqq{} zwei Funktionen von Mobilgeräten. Verwendet wird unter anderem die \textbf{Kamera} und \textbf{GPS-Sensorik} von heutigen Mobilgeräten.

\section{Projektteilnehmer\label{sec1.3:Unterpunkt-3}}

Das Projekt wird von sechs Student*innen bearbeitet. Zusätzlich wird das Projekt während der kompletten Laufzeit von einem Stakeholder (Dozent des Moduls) betreut.

\begin{table}[H]
    \centering
    \begin{tabular}{|l|l|l|}
    \hline
    \multicolumn{1}{|c|}{\textbf{Rolle}} & \multicolumn{1}{c|}{\textbf{Name}} & \multicolumn{1}{c|}{\textbf{Kontakt}} \\ \hline
    Mitglied                           & Benita Dietrich                    & i18008\myat hb.dhbw-stuttgart.de      \\ \hline
    Mitglied                           & Paul Finkbeiner                    & i18011\myat hb.dhbw-stuttgart.de      \\ \hline
    Mitglied                           & Josua Stricker                     & i18039\myat hb.dhbw-stuttgart.de      \\ \hline
    Mitglied                           & Jonas Schwarz                      & i18037\myat hb.dhbw-stuttgart.de      \\ \hline
    Mitglied                           & Sven Stoll                         & i18038\myat hb.dhbw-stuttgart.de      \\ \hline
    Mitglied                           & Moris Kotsch                       & i18021\myat hb.dhbw-stuttgart.de      \\ \hline
    Stakeholder                        & Torsten Hopf                       & torsten.hopf\myat mhp.com             \\ \hline
    \end{tabular}
\end{table}